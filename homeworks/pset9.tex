\newpage\documentclass[11pt,onecolumn,fleqn]{article}

\usepackage[latin1]{inputenc}
\usepackage{enumerate}
\usepackage[hang,flushmargin]{footmisc}
\usepackage{amsmath}
\usepackage{amsfonts}
\usepackage{amssymb}
\usepackage{amsthm}
\usepackage{xcolor}

\theoremstyle{definition}
\newtheorem{theorem}{Theorem}[section]
\newtheorem{lemma}[theorem]{Lemma}
\newtheorem{corollary}[theorem]{Corollary}
\newtheorem{proposition}[theorem]{Proposition}
\newtheorem{definition}[theorem]{Definition}
\newtheorem{example}[theorem]{Example}

\setlength{\oddsidemargin}{0px}
\setlength{\textwidth}{460px}
\setlength{\voffset}{-1.5cm}
\setlength{\textheight}{20cm}
\setlength{\parindent}{0px}
\setlength{\parskip}{10pt}

\newcommand{\TT}{$\checkmark$}
\newcommand{\FF}{$\times$}


\begin{document}
\thispagestyle{empty}
\begin{center}
{\Huge 21-128 and 15-151 problem sheet 9}

Solutions to the following seven exercises and optional bonus problem are to be submitted through 
gradescope by 11PM on

\textbf{Tuesday 22th November 2022.}

\end{center}


\subsubsection*{Problem 1}
By counting in two ways, prove that $n^2 = 2\binom{n}{2}+n$ for all $n \ge 0$.


\subsubsection*{Problem 2}
By counting in two ways, prove that $${n \choose j}{n \choose k} = \sum_{i = 0}^{\min(j,k)}{n \choose i}{n-i \choose j-i}{n-j \choose k-i}$$ for all $n,j,k \in \mathbb{N},\ j,k \leq n$.


\subsubsection*{Problem 3}
By counting in two ways, prove that $\sum_{i=1}^n (i-1)(n-i) = \binom{n}{3}$ for all $n \ge 1$.


\subsubsection*{Problem 4}
Let $x,y,z$ be nonnegative real numbers such that $y+z \ge 2$. Prove that $$(x+y+z)^2 \ge 4x+4yz$$


\subsubsection*{Problem 5} 
Consider the following system of equations of real numbers:
\begin{equation*}
    \begin{cases}
      3w + 2x + y + z = 14\\
      w^2 + x^2 + y^2 + z^2 = 14
    \end{cases}
\end{equation*}
What is the maximum possible value of $z$?


\subsubsection*{Problem 6}
The standard way to define ordered fields 
is to start with a strict order on $\mathbb{F}$ and then axiomatize the properties that 
make it compatible with arithmetic:
%
\begin{alignat*}{2}
       &(O1)   \qquad &     x < y  &\implies  x + z < y + z  \\[1ex]
       &(O2)   \qquad &     0 < x,y  &\implies  0 < x*y  
\end{alignat*}
%
Alternatively, we can introduce  positive sets $P \subseteq \mathbb{F}$ and use
them to define order:
%
\begin{alignat*}{2}
       &(P1)   \qquad &     x, y \in P  &\implies  x + y \in P  \\[1ex]
       &(P2)   \qquad &     x, y \in P  &\implies  x * y \in P  \\[1ex]
       &(P3)   \qquad &     x \in P \lor \phantom{ }&x = 0 \lor -x \in P 
\end{alignat*}
%
In $(P3)$, exactly one of the cases is supposed to hold. 
Given $<$ we can define $P_{<} = \{ x \in \mathbb{F} \mid  0 < x\}$ and, conversely, 
$x <_{P} y \Longleftrightarrow y-x \in P$.
\vspace{2ex} 

(a)  Show that $P_{<}$ is a positive set in any ordered field.

(b)  Show that for any positive set $P$, the order $<_{P}$ produces an ordered field.




\subsubsection*{Bonus}
Show by counting in two ways that: 
$$2^{(n^2)} = \sum_{i = 0}^n {n \choose i}(2^n - 1)^{i}$$



\end{document}
