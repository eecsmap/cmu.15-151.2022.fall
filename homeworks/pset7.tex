\newpage\documentclass[11pt,onecolumn,fleqn]{article}

\usepackage[latin1]{inputenc}
\usepackage{enumerate}
\usepackage[hang,flushmargin]{footmisc}
\usepackage{amsmath}
\usepackage{amsfonts}
\usepackage{amssymb}
\usepackage{amsthm}
\usepackage{xcolor}

\theoremstyle{definition}
\newtheorem{theorem}{Theorem}[section]
\newtheorem{lemma}[theorem]{Lemma}
\newtheorem{corollary}[theorem]{Corollary}
\newtheorem{proposition}[theorem]{Proposition}
\newtheorem{definition}[theorem]{Definition}
\newtheorem{example}[theorem]{Example}

\setlength{\oddsidemargin}{0px}
\setlength{\textwidth}{460px}
\setlength{\voffset}{-1.5cm}
\setlength{\textheight}{20cm}
\setlength{\parindent}{0px}
\setlength{\parskip}{10pt}

\newcommand{\TT}{$\checkmark$}
\newcommand{\FF}{$\times$}


\begin{document}
\thispagestyle{empty}
\begin{center}
{\Huge 21-128 and 15-151 problem sheet 7}

Solutions to the following seven exercises and optional bonus problem are to be submitted through 
gradescope by 11PM on

\textbf{Wednesday 9th November 2022.}

\end{center}

\subsubsection*{Problem 1}
For $x,y,z \in \mathbb{Z}$, suppose that $5$ divides $x^2+y^2+z^2$. Prove that $5$ divides at least one of $x$, $y$ or $z$.

\noindent
Hint: What remainders can be left when the square of an integer is divided by 5?


\subsubsection*{Problem 2}
The base $10$ representation of an integer is \textit{palindromic} if the digits read the same when written forward or backward. 
Prove that every palindromic integer with an even number of digits is divisible by $11$. 


\subsubsection*{Problem 3} 
Show your work in the following computations.
\begin{enumerate}[(a)]
    \item Determine the last two digits of $14^{2022}$.
    \item Compute $\frac{53!}{27} \bmod 27$.
    \item Find all integers $x$ such that $x^2 + 3x \equiv 3^{31} \mod 29$.
\end{enumerate}


\subsubsection*{Problem 4}
Show that the equation $x^{2}+1\equiv0\pmod{p}$ has a solution when $p$ prime and $p\equiv1\pmod{4}$. 

Hint: Wilson's Theorem.


\subsubsection*{Problem 5}
Let $m$ and $n$ be positive, relatively prime integers, and $r$ and $s$ be integers such that 
$mr \equiv 1 \bmod n$ and $ns \equiv 1 \bmod m$. For integers $a,b$, find an integer value of 
$x$ in terms of $a,b,m,n,r,s$ satisfying $x \equiv a \bmod n$ and $x \equiv b \bmod m$.


\subsubsection*{Problem 6} 
Let $A\subseteq\mathbb{N}^+$ and $B\subseteq\mathbb{N}^+$ be nonempty sets of positive integers.  
Define $$A + B \stackrel{\text{def}}{=} \{a+b:a\in A,b\in B\}.$$ 
Show that $A+B$ is finite if and only if both $A$ and $B$ are finite.

\subsubsection*{Problem 7}
For arbitrary $f : \mathbb{N} \rightarrow \mathbb{N}$ and $g : \mathbb{N} \rightarrow \mathbb{N}$, show that if the image of $g$ 
is finite, then the image of $f \circ g$ is finite with size less than or equal to size of the image of $g$.


\subsubsection*{Bonus Problem - 2 points}
Find all positive integers $a$ for which there exist non-negative integers $x_0, x_1,\dots x_{2020}$ satisfying the equation
$$a^{x_0} = a^{x_1} + a^{x_2} + \cdots + a^{x_{2020}}.$$


\end{document}
