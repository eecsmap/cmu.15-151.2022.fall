\documentclass[11pt,onecolumn,fleqn]{article}

\usepackage[latin1]{inputenc}
\usepackage{enumerate}
\usepackage[hang,flushmargin]{footmisc}
\usepackage{amsmath}
\usepackage{amsfonts}
\usepackage{amssymb}
\usepackage{amsthm}
\usepackage{xcolor}


\theoremstyle{definition}
\newtheorem{theorem}{Theorem}[section]
\newtheorem{lemma}[theorem]{Lemma}
\newtheorem{corollary}[theorem]{Corollary}
\newtheorem{proposition}[theorem]{Proposition}
\newtheorem{definition}[theorem]{Definition}
\newtheorem{example}[theorem]{Example}

\setlength{\oddsidemargin}{0pt}
\setlength{\textwidth}{460pt}
\setlength{\voffset}{-1cm}
\setlength{\textheight}{21cm}
\setlength{\parindent}{0pt}
\setlength{\parskip}{10pt}

\newcommand{\TT}{$\checkmark$}
\newcommand{\FF}{$\times$}

%---------------------------------------------------------------------------
% custom
%
\usepackage{mathtools}
\DeclarePairedDelimiter\card{|}{|}

\def\fall#1{\forall\, {#1}\;}
\def\fex#1{\exists \,{#1}\;}
\def\fexone#1{\exists_{1} \,{#1}\;}
\def\fmlmat#1{\bigl(\,{#1}\,\bigr)}

\def\set#1#2{\{\, {#1} \mid {#2}\,\}}


%---------------------------------------------------------------------------
\begin{document}
\thispagestyle{empty}
\begin{center}
{\Huge
21-128 and 15-151 problem sheet 2
}

Solutions to the following seven exercises and optional bonus problem are to be submitted through 
gradescope by 11PM on

\textbf{Wednesday 14th September 2022.}

\end{center}


\subsubsection*{Problem 1}
Determine which of the following assertions are true, where $A$, $B$ and $C$ are non-empty subsets of $\mathbb{Z}$.
\begin{enumerate}[(a)]
  \item $(A\cap B\cap C) \subset (A\cup B)$
  \item $(A\setminus B) \cap (B\setminus A) = \varnothing $
  \item $(A \cap B\neq \varnothing)\Longrightarrow ((A \setminus B)\subset A)$
\end{enumerate}

\subsubsection*{Problem 2}
Prove that
$$\{ x \in \mathbb{Z} : 5 \mid x \} = \{ x \in \mathbb{Z} : 5 \mid (10 - 4x) \}.$$

\subsubsection*{Problem 3}
Let $A, B \subseteq \mathbb{N}$ be finite and let 
$C = \set{a \in A}{ \fex{b \in B} ( a + b \in A )}$.\\ For what sets $A$, $B$ does $A = C$?

\subsubsection*{Problem 4}
Let $X \triangle Y = (X \setminus Y) \cup (Y \setminus X)$. Prove $$(A \triangle B) \setminus (B \triangle C) \subseteq ((B \cap C) \setminus A) \cup (A \setminus (B \cup C)). $$\\

\subsubsection*{Problem 5}
Find a non-empty set $A$ such that 
\begin{enumerate}
\item[1.] $\fall{S \in A} ( S \subseteq \mathbb{N} )$
\item[2.] $\fall{S \in A} \fex{S' \in A}\,\bigl(( S \neq S') \wedge (S \cup S') = S \bigr)$
\end{enumerate}

\subsubsection*{Problem 6}
Let $\mathbb{N^{+}}$ denote the set of positive integers and consider
the function $f : \mathbb{N^{+}} \times \mathbb{N^{+}} \to \mathbb{R}$ defined by
$$f(a,b) = \frac{(a+1)(a+2b)}{2}$$
\begin{enumerate}[(a)]
\item Show that the image of $f$ is a subset of $\mathbb{N^{+}}$.
\item Determine exactly which positive integers are elements of the image of $f$. 
(\textbf{Hint:} Formulate a hypothesis by trying values.)
\end{enumerate}

\subsubsection*{Problem 7}
For $a \in \mathbb{R}$ and $f : \mathbb{R} \to \mathbb{R}$, show that (a) and (b) below have different meanings.
\begin{enumerate}[(a)]
\item $\fall{ \varepsilon > 0} \fex{\delta > 0} \fall{x \in \mathbb{R}} 
\bigl( \card{x-a} < \delta \Rightarrow \card{f(x) - f(a)} < \varepsilon \bigr)$.
\item $\fex{\delta > 0} \fall{\varepsilon > 0} \fall{x \in \mathbb{R}} 
\bigl( \card{x-a} < \delta \Rightarrow \card{f(x) - f(a)} < \varepsilon \bigr)$.
\end{enumerate}
(\textbf{Hint:} Find a function $f : \mathbb{R} \to \mathbb{R}$ and an element $a \in \mathbb{R}$ for which 
(a) and (b) have different truth values.)

\subsubsection*{Bonus Problem  (2 points)}
A function $g : \mathbb{R} \to \mathbb{R}$ is \textit{even} if $g(-x) = g(x)$ for all $x \in \mathbb{R}$, 
or \textit{odd} if $h(-x)=-h(x)$ for all $x \in \mathbb{R}$.

Let $f : \mathbb{R} \to \mathbb{R}$.
\begin{enumerate}[(a)] \vspace{-10pt}
\item Prove that there exists a unique pair of functions $g,h : \mathbb{R} \to \mathbb{R}$ such that $g$ is even, 
$h$ is odd, and $f=g+h$. (\textbf{Hint:} Express both $f(x)$ and $f(-x)$ in terms of $g(x)$ and $h(x)$, and solve 
the resulting system of equations.)
\item When $f$ is a polynomial function, express $g$ and $h$ as in (a) in terms of the coefficients of $f$.
\end{enumerate}


\end{document}

