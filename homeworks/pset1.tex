\documentclass[11pt,onecolumn,fleqn]{article}

\usepackage[latin1]{inputenc}
\usepackage{enumerate}
\usepackage[hang,flushmargin]{footmisc}
\usepackage{amsmath}
\usepackage{amsfonts}
\usepackage{amssymb}
\usepackage{amsthm}
\usepackage{xcolor}

\theoremstyle{definition}
\newtheorem{theorem}{Theorem}[section]
\newtheorem{lemma}[theorem]{Lemma}
\newtheorem{corollary}[theorem]{Corollary}
\newtheorem{proposition}[theorem]{Proposition}
\newtheorem{definition}[theorem]{Definition}
\newtheorem{example}[theorem]{Example}

\setlength{\oddsidemargin}{0pt}
\setlength{\textwidth}{460pt}
\setlength{\voffset}{-1cm}
\setlength{\textheight}{21cm}
\setlength{\parindent}{0pt}
\setlength{\parskip}{10pt}

\newcommand{\TT}{$\checkmark$}
\newcommand{\FF}{$\times$}

\def\fall#1{\forall\, {#1}\,}
\def\fex#1{\exists \,{#1}\,}
\def\fexone#1{\exists_{1} \,{#1}\,}
\def\fmlmat#1{\bigl(\,{#1}\,\bigr)}


\begin{document}
\pagestyle{empty}
\begin{center}
{\Huge
21-128 and 15-151 problem sheet 1
}

Solutions to the following seven exercises and optional bonus problem are to be submitted through 
gradescope by 11PM on

\textbf{Wednesday 7th September 2022.}

\end{center}


\subsubsection*{Problem 1}
For each statement below, decide whether it is true or false. Prove your claim using only properties 
of the natural numbers.
\begin{enumerate}[(a)] \vspace{-10pt}
\item If $n \in \mathbb{N}$ and $n^2+(n+1)^2 = (n+2)^2$, then $n=3$.
\item For all $n \in \mathbb{N}$, it is false that $(n-1)^3 + n^3 = (n+1)^3$.
\end{enumerate}


\subsubsection*{Problem 2}
\begin{enumerate}[(a)]
    \item Show that $( p \Rightarrow q ) \vee ( p \Rightarrow r )$ and $p \Rightarrow ( q \vee r )$ are logically equivalent. 
    \item Show that $(\fall{x} \in S \ P(x)) \vee (\fall{x} \in S \  Q(x))$ and $\fall{x} \in S \  ( P(x) \vee Q(x) )$\\ are not logically equivalent, where $P(x)$ and $Q(x)$ are logical formulae and $S$ is a set.
\end{enumerate}


\subsubsection*{Problem 3}
Let $p(x,y)$ be the predicate `$x + y$ is even', where $x$ and $y$ range over the integers.
\begin{enumerate}[(a)]
\item Prove that $\fall{x} \fex{y}\, p(x,y)$ is true.
\item Prove that $\fex{y}\fall{x}\,  p(x,y)$ is false.
\end{enumerate}


\subsubsection*{Problem 4}
You have $m$ indistinguishable marbles, $m \in \mathbb{Z}^+$, and 5 indistinguishable bags.
\begin{enumerate}[(a)]
\item What is the smallest number of marbles such that you are guaranteed to have 5 marbles in the same bag or 5 different bags with at least one marble? Give proof.
\item Suppose there are $n$ marbles and $k$ bags, where $n$ and $k$ are positive integers. Prove that there is a unique way to distribute the marbles such that the number of marbles in each bag differs by at most one.
\end{enumerate}


\subsubsection*{Problem 5}
A Pythagorean triple is a triple of positive integers $(a,b,c)$ such that $a^2 + b^2 = c^2$. Let $(x,y,z)$ be a Pythagorean triple, and let $P=x+y+z$ and $A=\frac{1}{2}xy$ be the perimeter and area, 
respectively, of the right-angled triangle whose side lengths are $x$, $y$ and $z$.
\begin{enumerate}[(a)]
\item Find the possible values of $(x,y,z)$ when $P=A$.
\item Find the possible values of $(x,y,z)$ when $P=2A$.
\end{enumerate}

\subsubsection*{Problem 6}
\begin{enumerate}[(a)]
\item Show that the following statement is false:
$$\text{For all } a,x \in \mathbb{R} \text{ there is a unique } y \in \mathbb{R} \text{ such that } x^4y+ay+x=0$$
\item Find the set of real numbers $a$ such that the following statement is true:
$$\text{For all } x \in \mathbb{R} \text{ there is a unique } y \in \mathbb{R} \text{ such that } x^4y+ay+x=0$$
\end{enumerate}


\subsubsection*{Problem 7}
Which of the following numbers are irrational for every choice of numbers $r$, $a$ and $b$, such that
$r$ is rational and $a$ and $b$ are irrational?
$$a+r \qquad a+b \qquad ar \qquad ab \qquad a^r \qquad r^a \qquad a^b$$
Prove your claims, either by proving that the number must always be irrational or by providing a counterexample. 
If you claim that a number is irrational, then you should prove it.


\subsubsection*{Bonus Problem (2 points)}
Three brilliant, flawless logicians - Aimee, Brad, and Cindy were blindfolded and each had a hat with a 
positive integer (possibly different for each) written on it placed on their heads.

Their blindfolds were then removed; they faced each other in a circle and each could see the 
hats the others were wearing, but not their own hat.

They were told that two of the numbers added up to the third. In order to be generously rewarded 
they needed to figure out what number was written on their hats.

Here is the conversation that took place:

Aimee: I don't know what my number is.
\newline
Brad: I don't know what my number is.
\newline
Cindy: I don't know what my number is.
\newline
Aimee: Now I know what my number is. It is 50.

\begin{enumerate}[(a)]
	\item What are the other numbers?
	\item What combination(s) of numbers would allow Cindy to solve the problem in round 1?
\end{enumerate}


\end{document}
