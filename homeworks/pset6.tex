\newpage\documentclass[11pt,onecolumn,fleqn]{article}

\usepackage[latin1]{inputenc}
\usepackage{enumerate}
\usepackage[hang,flushmargin]{footmisc}
\usepackage{amsmath}
\usepackage{amsfonts}
\usepackage{amssymb}
\usepackage{amsthm}
\usepackage{xcolor}

\theoremstyle{definition}
\newtheorem{theorem}{Theorem}[section]
\newtheorem{lemma}[theorem]{Lemma}
\newtheorem{corollary}[theorem]{Corollary}
\newtheorem{proposition}[theorem]{Proposition}
\newtheorem{definition}[theorem]{Definition}
\newtheorem{example}[theorem]{Example}

\setlength{\oddsidemargin}{0px}
\setlength{\textwidth}{460px}
\setlength{\voffset}{-1.5cm}
\setlength{\textheight}{20cm}
\setlength{\parindent}{0px}
\setlength{\parskip}{10pt}

\newcommand{\TT}{$\checkmark$}
\newcommand{\FF}{$\times$}


\begin{document}
\thispagestyle{empty}
\begin{center}
{\Huge 21-128 and 15-151 problem sheet 6}

Solutions to the following two exercises and optional bonus problem are to be submitted through 
gradescope by 11PM on

\textbf{Wednesday 26th October 2022.}

\end{center}

\subsubsection*{Problem 1}
Rose, Emily, Susan, Micah, and Harrison are sailors on a tropical island. They spend the day gathering a pile of coconuts. Exhausted, they postpone dividing it until the next morning. Suspicious, each decides to take their share during the night. Rose divides the pile into 
five equal portions plus one extra coconut, which she gives to a monkey. She takes one pile and leaves the rest in 
a single pile. Emily later does the same; again the monkey receives one leftover coconut. Susan, Micah, and Harrison all do this; each time, a remainder of one goes to the monkey. In the morning, they split 
the remaining coconuts into five equal piles, and each sailor gets their ``share''. (Each knows some were taken, but 
none complains, since each is guilty!) What is the smallest number of coconuts in the original pile?

\noindent
Hint: The recursive calculations of the number of coconuts remaining are substantially simplified by assuming that 
there are $n-4$ coconuts initially.

\subsubsection*{Problem 2}
Let $p$ be a prime number greater than 2. Prove that $\frac{2}{p}$
can be expressed in exactly one way in the form $\frac1m + \frac1n$ where $m$ and $n$ are positive
integers with $n > m$. 

\noindent
Hint: For proving uniqueness, try to get the equation in a form where you can case on whether $p \mid n$ or $p \mid m$.


\subsubsection*{Bonus Problem - 2 points}
A census taker interviews a woman in a house. ``Who lives here?'' he asks. ``My husband and I and my three daughters,'' she replies. ``What are the ages
of your daughters?'' ``The product of their ages is 36 and the sum of their ages is the house number.'' The census taker looks at the house number, thinks,
and says, ``You haven't given me enough information to determine the ages.'' ``Oh, you're right,'' she replies, ``Let me also say that my eldest daughter is asleep
upstairs.'' ``Ah! Thank you very much!'' What are the ages of the daughters? (The problem requires ``reasonable'' mathematical interpretations of its words.)


\end{document}
