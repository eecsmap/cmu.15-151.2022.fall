\newpage\documentclass[11pt,onecolumn,fleqn]{article}

\usepackage[latin1]{inputenc}
\usepackage{enumerate}
\usepackage[hang,flushmargin]{footmisc}
\usepackage{amsmath}
\usepackage{amsfonts}
\usepackage{amssymb}
\usepackage{amsthm}
\usepackage{xcolor}

\theoremstyle{definition}
\newtheorem{theorem}{Theorem}[section]
\newtheorem{lemma}[theorem]{Lemma}
\newtheorem{corollary}[theorem]{Corollary}
\newtheorem{proposition}[theorem]{Proposition}
\newtheorem{definition}[theorem]{Definition}
\newtheorem{example}[theorem]{Example}

\setlength{\oddsidemargin}{0px}
\setlength{\textwidth}{460px}
\setlength{\voffset}{-1.5cm}
\setlength{\textheight}{20cm}
\setlength{\parindent}{0px}
\setlength{\parskip}{10pt}

\newcommand{\TT}{$\checkmark$}
\newcommand{\FF}{$\times$}


\begin{document}
\thispagestyle{empty}
\begin{center}
{\Huge 21-128 and 15-151 problem sheet 8}

Solutions to the following seven exercises and optional bonus problem are to be submitted through 
gradescope by 11PM on

\textbf{Wednesday 16th November 2022.}

\end{center}

\textbf{Note:} You don't need to simplify your answers. Sums and products of integers, binomial coefficients, exponents, factorials, etc. are acceptable.

\subsubsection*{Problem 1}
How many ways are there to pick two cards from a standard $52$-card deck, without replacement, such that the first card is spades and the second card is not a face card?


\subsubsection*{Problem 2}
Count the number of 6 card hands where each suit in the hand occurs at least twice AND each rank in the hand is unique.


\subsubsection*{Problem 3}
Find the number of functions $f: [6] \to [6]$ such that $f$ contains exactly three elements in its image.


\subsubsection*{Problem 4}
Theo wants to show his appreciation to the TAs for their hard work this semester, so he goes to the 151/128 store to buy them gifts! Note that there are 28 TAs on staff this year.

He buys 100 hedgehog plushies and 100 bottles of Diet Coke. He wants to give away all of the gifts and make sure each TA gets a least one gift. But, no TA should receive both plushies and Diet Coke. How many ways can Theo do this?

\newpage

\subsubsection*{Problem 5}
Consider a set of $1\times 1$ squares and $1\times 2$ dominoes.  The $1\times 1$ squares are colored either red or blue; 
the $1\times 2$ dominoes are colored green, yellow, or black.  Let $A_n$ denote the number of ways to tile a $1\times n$ 
board of unit squares with these squares and dominoes such that no two pieces overlap.

\begin{enumerate}

\item Show that $A_n = 2A_{n-1} + 3A_{n-2}$ for all $n\geq 2$.

\item Deduce that $$A_n = \dfrac{3^{n+1} + (-1)^n}4\quad\text{for all }n\geq 1.$$

\end{enumerate}


\subsubsection*{Problem 6}
How many nine-digit integers consisting of each of the digits 1-9 exactly once are divisible by $36$?


\subsubsection*{Problem 7}
By counting in two ways, prove that $\sum_{k=1}^n 2^{k-1} = 2^n - 1$ for all $n \in \mathbb{N}$.


\subsubsection*{Bonus Problem - (2 points)} Tram tickets have six-digit numbers (from $000000$ to $999999$). A ticket 
is called \textit{lucky} if the sum of its first three digits is equal to the sum of its last three digits. 
A ticket is called \textit{medium} if the sum of all its digits is 27. Let $A$ and $B$ denote the numbers of lucky 
tickets and medium tickets respectively. Prove that $A=B$. 

\end{document}
